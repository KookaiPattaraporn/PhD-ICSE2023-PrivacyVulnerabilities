\section{Related Work} \label{sec:related-work}

Several systems have been established to standardise the reporting process and structure of common vulnerabilities (e.g. CWE \cite{CWE}, CVE \cite{CVE} and OWASP \cite{OWASP2020}). However, identifying the root causes of the reported vulnerabilities is still a time-consuming and expertise-required process \cite{Gonzalez2019}. Recent work employed information retrieval, data mining, natural language processing, machine learning and deep learning techniques to characterise vulnerabilities reported in CVE and CWE systems \cite{Li2017, Gonzalez2019, Liu2020a}. \newtext{\citeauthor{Li2017} developed a vulnerability mining algorithm to characterise software vulnerabilities \cite{Li2017}. It first created a Vulnerability Knowledge Discovery Database (VKDD) by extracting content from CWE, CVE and National Vulnerability databases. It then employed the semantic model to select terms that are relevant to software vulnerabilities from the VKDD. The association and classification rules were later determined to identify and classify the vulnerabilities. \citeauthor{Liu2020a} built a classification model for detecting vulnerable functions in source code \cite{Liu2020a}. The deep neural network with bidirectional long short-term memory (LSTM) was employed to learn high-level representations of the program's abstract syntax trees. The study also proposed a fuzzy-based oversampling method to mitigate the class imbalance between vulnerable and nonvulnerable code.} However, privacy vulnerabilities were not addressed in those previous work.

Recent work have also studied and made use of the CWE and CVE systems. For example, the work in \cite{Bhandari2021} collected CVE records with their associated CWEs and code commits. The collected information was then analysed to produce insightful metadata such as concerned programming language and code-related metrics. This work can be applied in multiple applications related to software maintenance such as automated vulnerability detection and classification, vulnerability fixing patches analysis and program repair. \citeauthor{Galhardo2020} proposed a formulation to calculate the most dangerous software errors in CWE \cite{Galhardo2020}. They used this formulation to identify the top 20 most significant CWE records in 2019. Again, these prior work only focuses on security vulnerabilities.

\citeauthor{Yang2013} introduced a framework to detect privacy leakage in mobile applications \cite{Yang2013}. This study identified several common privacy vulnerabilities in Android such as unintended sensitive data transmission and local logging. \citeauthor{Ma2013} discussed a privacy vulnerability in mobile sensing networks which collect mobility traces of people and vehicles (e.g. traffic monitoring) \cite{Ma2013}. Although these networks receive anonymous data, it was proven in the study that these data can identify victims. These studies have confirmed the occurrence of privacy vulnerabilities in multiple types of software systems (e.g. web/mobile applications and sensing networks). However, most of the existing studies only focused on security concerns when investigating software vulnerabilities, thus overlooked privacy-related concerns in many contexts.

A number of studies have proposed state-of-the-art taxonomies of privacy threats (e.g. \cite{Solove2006a, Deng2011, Wuyts2014a}). \citeauthor{Solove2006a} proposed four categories of privacy threats which cover harmful activities that can violate privacy of individuals \cite{Solove2006a}. Another well-known standardised taxonomy of privacy threats in software known as LINDDUN was first proposed in \cite{Deng2011}. LINDDUN is a model-based technique used to discover the privacy threats in a system \cite{Wuyts2014a}. It models the system as a data flow diagram (DFD), maps DFD elements to privacy threat categories, and then identifies and documents the privacy threat type and the DFD element type. Although LINDDUN can be applied to any general-purpose software systems, it requires the details of system description to construct the DFD for threat analysis.

\newtext{Two renowned organisations have also worked on privacy framework and privacy threat modelling. National Institute of Standards and Technology (NIST) proposed a privacy framework which focuses mostly on providing high-level guidelines that organisations can follow to govern and control privacy risks \cite{NIST2020}. It does not address the privacy threats specifically in software systems. Thus, we did not include this framework in our study. The MITRE Corporation has been developing a privacy threat modelling method \cite{Bloom2022}. However, the MITRE work took a different approach: they identified privacy attacks from the Federal Trade Commission (FTC) and Federal Communications Commission (FCC) closed cases. By contrast, we identified privacy threats from existing privacy engineering literature, privacy regulations/frameworks, and industry sources. This results in several major differences in the two taxonomies in terms of scope of privacy threats, coverage of privacy threats, and classification of threats and (sub-)categories. In addition, our work goes even beyond providing a taxonomy. We provided a detailed, concrete (CWE-ready) description of the common privacy vulnerabilities including how they occur and their consequences, how to detect them and mitigate them, and demonstrative examples. These provide software engineers with more concrete information of privacy vulnerabilities.}