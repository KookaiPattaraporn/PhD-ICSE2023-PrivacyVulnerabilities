\section{Introduction} \label{sec:intro}

Technologies and digitalisation are rapidly emerging in our society. People significantly rely on software applications and computing devices in their daily lives, consequently leaving their traceable digital activities, contributions and communications on those digital devices across the Internet \cite{Statista2021b}. Even when people are not using software, data about their normal life activities may also be collected by software applications through the ubiquity of IoT and GPS devices, surveillance cameras, face recognition apps and so on. Thus, our privacy is under constant threat in this current digital age. In fact, privacy invasions and attacks have been increasing significantly in recent years \cite{Statista, OAIC2021}. For examples, a cyber crime in the U.S. in 2018 exposed 471 million personal records, and a breach of a national ID database in India leaked over 1.1 billion records including biometric information (e.g. iris and fingerprint scans) \cite{Statista2021}. Those incidents and threats raise an urgent need for privacy to be deeply integrated into the development, testing and maintenance of software applications.

Although security and privacy are often discussed together, they are \emph{not} the same \cite{Bambauer2013}. Security often refers to protection against the unauthorised access to software applications and the data they collect and store. On the other hand, privacy relates to protection of the individual rights to their personally identifiable information in terms of how those personal data are collected, used, protected, transferred, altered, disclosed and destroyed \cite{ISO/IEC2011}. For example, security controls are put in place to ensure that only people with credentials have access to a software application in a hospital. However, if anyone with valid credentials can see patient health records using this software, then privacy is not protected. This example demonstrates that security can be achieved without privacy, however security is an essential component for privacy protection.

Cyberattacks, either in the form of security or privacy attacks, are often formed by exploiting \emph{vulnerabilities} or \emph{weaknesses}\footnote{The two terms are often interchangeable. Hereby, we will use vulnerabilities to refer to both of them.} found in software systems. For instance, the infamous WannaCry ransomware attack exploited a vulnerability in Microsoft Windows systems, while the Heartbleed vulnerability in OpenSSL has made millions of websites and online platforms across the world vulnerable to cyberattacks. To prevent similar attacks, efforts have been put into understanding and publicly disclosing vulnerabilities so that developers can identify and fix them in their software applications. These efforts have resulted in the widely-known CWE and CVE systems \cite{CWE, CVE}.

However, there have been very little work (e.g. \cite{Yang2013, Ma2013}) in identifying privacy vulnerabilities. A system which specifically records common privacy vulnerabilities does \emph{not} exist yet. Thus, software developers often rely on the CWE and CVE systems to learn about privacy-related weaknesses and vulnerabilities. However, it is not clear to what extent privacy concerns are covered in those systems, and whether privacy receives adequate attention (which it deserves). To answer these questions, we have collected \emph{all} 922 weaknesses recorded in CWE and 156,537 records registered in CVE to date, filtered out non privacy-related records and further analysed the shortlisted records that are privacy-related. We have found only 41 and 157 privacy vulnerabilities in the CWE and CVE systems respectively. The coverage of privacy-related vulnerabilities in both systems is very limited, only 4.45\% in CWE and 0.1\% in CVE. (\textbf{\underline{Contribution 1}}) %To answer these questions, we have analysed \emph{all} 922 weaknesses recorded in CWE and 156,537 records registered in CVE to date.

The next questions we aimed to explore are what privacy threats are covered in those privacy-related vulnerabilities in the CWE/CVE systems and if they are adequately cover the privacy threats raised in both research and practice. To answer these questions, we have conducted an explanatory study on the privacy engineering literature, privacy standards and frameworks (e.g. ISO/IEC 29100), regulations in different jurisdiction, including the European Union General Data Protection Regulation (EU GDPR), California Consumer Privacy Act (CCPA), Health Insurance Portability and Accountability Act (HIPAA), Gramm-Leach Bliley Act (GLBA), the U.S. Privacy Act (USPA) and the Australian Privacy Act (APA), and relevant reputable organisations (e.g. OWASP \cite{OWASP2020} and Norton \cite{Nortona}). This explanatory study informed us to develop a taxonomy of common privacy threats that have been raised in research and practice. The taxonomy is built upon the existing well-known privacy threats taxonomy \cite{Stallings2019}. Multiple raters/coders then examined all 41 and 157 privacy vulnerabilities in the CWE and CVE systems, and mapped them to this taxonomy. The Cohen’s Kappa coefficient, used to measure the inter-rater agreement, was obtained at 0.874 and 0.875 for the CWEs and CVEs respectively, an almost perfect agreement, suggesting the strong reliability of the classification. We found that the existing privacy weaknesses and vulnerabilities reported in the CWE and CVE systems cover only 13 out of 24 common privacy vulnerabilities raised in research and practice. Many important types of privacy weaknesses and vulnerabilities are \emph{not} covered such as improper personal data collection, use and transfer, allowing unauthorised actors to modify personal data, processing personal data at third parties, and improper handling of user privacy preferences and consent. (\textbf{\underline{Contribution 2}})

These actionable insights led to our proposal of 11 new common privacy weaknesses to CWE\footnote{We chose CWE instead of CVE since CVEs specify unique vulnerabilities detected in specific software systems and application, while CWEs are at a more abstract, generic level.}. These new CWE entries cover the areas of privacy threats that have been raised in research and practice but do not exist in CWE yet. To further confirm the relevance and validity of our proposal, we extracted real code examples from software repositories that match with the new CWEs. Our contribution follows the CWE's true spirit of a community-developed list, and will enhance the CWE system to serve as a common language and baseline for identifying, mitigating and preventing not only security but also privacy weaknesses and vulnerabilities. (\textbf{\underline{Contribution 3}})

The remainder of the paper is structured as follows. Section \ref{sec:related-work} discusses related existing work in security and privacy vulnerabilities in software applications. The identification of privacy-related vulnerabilities in CWE and CVE is presented in Section \ref{sec:identifying-privacy-vul}. Section \ref{sec:common-privacy-concerns} discusses the taxonomy of privacy threats and how the privacy-related vulnerabilities in CWE and CVE systems cover those privacy threats. Section \ref{sec:cwe-proposal} presents a new common privacy weakness proposal. The threats to validity of our study are discussed in Section \ref{sec:threats}. We conclude and discuss future work in Section \ref{sec:conclusion}. Finally, we provide the details of a replication package and instructions on how to access it in Section \ref{sec:data-availability}.

%Software systems and applications have been attacked by various types of internal and external threats. The number of cyber attack incidents have been increasing for the last twelve years in different sectors (e.g. federal government authorities, businesses, healthcare providers and financial institutes) \cite{Purplesec2021}. Security vulnerability is one of the major weaknesses that make the software systems and applications expose to attacks.

%Those vulnerabilities can cause undesirable incidents in software systems and applications such as unauthorised access, modified data, system unavailability and data breaches. They do not only violate security principles, but they also expose user privacy in many cases. For examples, Yahoo hack, one of the largest data breaches, had 3 billion accounts been affected from the attack in 2017, a cyber crime in the U.S. in 2018 exposed 471 million personal records, and a breach of India's national ID database leaked over 1.1 billion records including biometric information (e.g. iris and fingerprint scans) \cite{Statista2021}. In addition, the Office of the Australian Information Commissioner (OAIC) has published a report revealing that 58\% of the notifiable data breaches between July and December 2020 were caused by malicious or criminal attacks \cite{OAIC2021}. These incidents affect a lot of individuals whose personal data was exposed to the public as well as the reputation of organisations. Thus, it is crucial to detect any security and privacy risks, weaknesses and vulnerabilities that can lead to possible attacks when maintaining software systems.

%Several studies have investigated security vulnerabilities in software systems and applications and proposed several methods and tools to help detect security weaknesses and vulnerabilities in codebases \cite{Li2018, Russell2019, Chakraborty2021, Bhandari2021, Zhan2021}. However, they did not consider the exposures of user privacy from those vulnerabilities. Thus, our paper aims to focus on privacy-related weaknesses and vulnerabilities in well-known and widely-used CWE and CVE. Privacy has gained an attention from software developers as they have to develop software systems that comply with concerned data protection and privacy regulations and frameworks. There are many regulations, standards and frameworks putting in place to govern the processing of personal data such as the European Union General Data Protection Regulation (EU GDPR). Hence, privacy-related vulnerability detection is significant to help interested parties focus on flaws and potential risks in software systems and applications that could affect privacy of individuals.


