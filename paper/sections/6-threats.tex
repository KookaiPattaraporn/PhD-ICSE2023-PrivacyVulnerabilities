\section{Threats to validity} \label{sec:threats}

\textit{\textbf{Internal validity.}} Our method for the extraction of privacy vulnerabilities in CWE and CVE using keywords might not result in the complete list. We also note that we did not consider the variations of the selected keywords (e.g. plural forms). However, we have used several strategies to mitigate these threats such as determining the keywords based on alternate terms described in CWE, and using the frequent terms identified in the studies that performed a large-scale analysis in privacy policies and considering general terms to cover unseen materials (e.g. regulation, data protection and privacy standard). 

\textit{\textbf{External validity.}} Our taxonomy of common privacy threats is constructed and refined based on the existing privacy threats taxonomy. We have put our best effort to ensure the comprehensiveness of the study by examining popular software engineering publication venues, well-established data protection regulations and privacy frameworks, and relevant reputable organisations. However, we acknowledge that there might be other sources of other privacy threats that we have not identified yet. We have carefully defined a set of inclusion and exclusion criteria to select the most relevant papers so that we got a reasonable number of papers to be examined individually. Future work would involve expanding our explanatory study to increase the generalisability of our taxonomy for common privacy threats. In addition, classifying the privacy-related vulnerabilities in CWE and CVE into the common privacy threats in the taxonomy involved subjective judgements. We have applied several strategies (e.g. using multiple coders, applying inter-rater reliability assessments and conducting disagreement resolution) to mitigate this threat. Future work could explore the use of external subject matter experts in these tasks.

